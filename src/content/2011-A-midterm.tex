\documentclass{paper}

\usepackage{xeCJK}
\usepackage{ifthen}

\usepackage{geometry}
	\geometry{left=1in,right=1in,top=1in,bottom=1in}

\usepackage{titling}

\usepackage{amsmath,amssymb}

\usepackage{tikz}
	\usetikzlibrary{calc}

\usepackage{environ}

\usepackage{enumitem}

\input{../environment.tex}
% 辅助
\newcommand{\nospace}{\unskip\ignorespaces}
\newcommand{\hideIf}[2]{\ifthenelse{#1}{\phantom{#2}}{#2}}

% 试卷环境
\NewEnviron{ExamPaper}{
	\begin{document}

	\newcounter{question}[section]
	\newcounter{option}[question]

	\maketitle
	\BODY

	\end{document}
}

\catcode`\。=13
\def。{.}

% 将 section 重定义为题目类别
\renewenvironment{section}[2][]{
	\\
	\stepcounter{section}
	\noindent \textbf{\Roman{section} #2} \quad #1
	\\
}{
	\\
}

% 题目
\newcommand{\question}[1]{
	\stepcounter{question}
	\noindent \textbf{\arabic{question}.} #1
	\\
}

% 答案,在 \noAnswer 为 true 时隐藏
\newcommand{\answer}[1]{\hideIf{\equal{\noAnswer}{true}}{\nospace #1\nospace}}
\NewEnviron{Answer}{
	\\
	\answer{\noindent 解\quad\BODY}
	\\
}

% 行内答案,有下划线
\newcommand{\inlineAnswer}[1]{\nospace\ \underline{\,\answer{#1}\,}\ \nospace}

% 选择题答案
\newcommand\option{
	\addtocounter{option}{1}
	(\Alph{option})
}
\newcommand{\options}[1]{
	\\
	\begin{tabular}{llll}
		#1
	\end{tabular}
	\\
}

% 右侧打分
\newcommand{\rightdot}[1]{\dotfill #1 \\}
\newcommand{\rightgrade}[1]{\rightdot{(#1 分)}}

% 数学
\newcommand{\ds}{\displaystyle}

\let\nvec\vec
\renewcommand{\vec}[1]{\ensuremath{\overrightarrow{\mathbf{#1}}}}

\let\der\partial
\renewcommand{\partial}[3][]{\frac{\der^{#1}#3}{\der{#2}^{#1}}}

\def\d{\mathrm d}
\def\D{\mathrm D}
\renewcommand{\derivative}[3][]{\frac{\d^{#1}#3}{\d{#2}^{#1}}}

\author{CUC Life Hack}
\date{}
\title{2011--2012 高等数学A 期中试卷}

\begin{ExamPaper}
	\section{单项选择题}

	\question{
		设非零向量 $\vec a$ 与 $\vec b$ 不平行,$\vec c=(\vec a\times\vec b)\times\vec a$,则\inlineAnswer{B}。
	}
	\options{
		\option $\vec c=\vec 0$ &
		\option $\ds\langle\vec b,\vec c\rangle <\frac\pi 2$ &
		\option $\vec c\perp\vec b$ &
		\option $\ds\langle\vec b,\vec c\rangle >\frac\pi 2$
	}
	
	\question{
		函数$f(x,y)=\left\{\begin{aligned}
			& \frac{xy}{\sqrt{x^2+y^2}}, & (x^2+y^2\neq 0) \\
			& 0 & (x^2+y^2=0)
		\end{aligned}\right.$\inlineAnswer{A}。
	}
	\options{
		\option 处处连续 &
		\option 处处有极限,但不连续 \\
		\option 仅在$(0,0)$点连续 &
		\option 除$(0,0)$点外处处连续 &
	}
	
	\question{
		设$u=f(r)$,而$r=\sqrt{x^2+y^2+z^2}$,$f(r)$具有二阶连续导数,则$
			\ds\partial[2]{x}{u}+\partial[2]{y}{u}+\partial[2]{z}{u}=
		$\inlineAnswer{B}。
	}
	\options{
		\option $\ds f''(r)+\frac1rf'(r)$ &
		\option $\ds f''(r)+\frac2rf'(r)$ \\
		\option $\ds\frac1{r^2}f''(r)+\frac1rf'(r)$ &
		\option $\ds\frac1{r^2}f''(r)+\frac2rf'(r)$ &
	}
	
	\question{
		若$f(x,2x)=x^2+3x,\ f_x'(x,2x)=6x+1$,则$f_y'(x,2x)=$\inlineAnswer{D}。
	}
	\options{
		\option $x+\frac32$ &
		\option $x-\frac32$ &
		\option $2x+1$ &
		\option $-2x+1$ &
	}

	\section{填空题}

	\question{
		已知点$A(3,1,-2)$和向量$\vec{AB}=\langle4,-3,1\rangle$,则$B$点的坐标为
		\inlineAnswer{$(7,-2,-1)$}。
	}
	\question{
		已知点$M_1(1,-1,2),\ M_2(3,3,1),\ M_3(3,1,3)$在平面$\pi$上,$\vec n$是$\pi$的单位法向量,且$\vec n$与$z$轴成锐角,则$\vec n=$\inlineAnswer{$
			\ds\frac1{\sqrt{17}}(-3,2,2)
		$}。
	}
	\question{
		设函数$z=z(x,y)$由方程$2x^2-3y^2+2z^2+6xy-14x-6y+z+4=0$确定,则函数$z$的驻点是\inlineAnswer{$(2,1)$}。
	}
	\question{
		曲面$x+xy+xyz=9$在点$(1,2,3)$处的切平面方程为\inlineAnswer{$9x+4y+2z=23$},法线方程为\inlineAnswer{$\ds\frac{x-1}9=\frac{y-2}4=\frac{z-3}2$}。
	}
	\question{
		曲面$\ds\frac{x^2}4+\frac{y^2}4-z^2=1$的名称是\inlineAnswer{旋转单叶双曲面},它是由$yOz$平面上的曲线$\ds\frac{y^2}4-z^2=1$绕$z$轴旋转产生的。
	}
	\question{
		已知$\ds e^{t^2},e^{-t^2}$是微分方程$\ds x''-\frac1tx'-4t^2x=0$的两个线性无关特解,则此方程的通解为\inlineAnswer{$\ds x=C_1e^{t^2}+C_2e^{-t^2}$},其中$C_1,C_2$为任意常数。
	}

	\section{解答题}

	\question{
		已知点$A(-3,1,6)$及点$B(1,5,-2)$,试在$yOz$面上求点$P$,使$|AP|=|BP|$,且点$P$到$Oy,Oz$轴等距离。
	}
	\begin{Answer}
		设点$P$为$(0,y,z)$。
		$$\begin{cases}
			\ds 9+(y-1)^2+(z-6)^2=1+(y-5)^2+(z+2)^2, \\
			|y|=|z|.
		\end{cases}$$
		\rightgrade{6}
		得$\ds y_1=z_1=2,\ -y_2=z_2=\frac23$。
		故所求点为$P_1(0,2,2)$或$\ds P_2=(0,-\frac23,\frac23)$。
		\rightgrade{10}
	\end{Answer}

	\question{
		过两点$M(0,4,-3)$和$N(6,4,-3)$作平面,使之不过原点,且使其在坐标轴上截距之和等于零,求此平面方程。
	}
	\begin{Answer}
		设平面方程为$\ds \frac xa+\frac yb-\frac z{a+b}=1$。
		\rightgrade{4}
		由过$M,N$点,则
		$$\begin{cases}
			\ds \frac4b+\frac3{a+b}=1, \\
			\ds \frac6a-\frac4b-\frac3{a+b}=1.
		\end{cases}$$
		解得$a=3,b=-2,6$。
		\rightgrade{7}
		故平面方程为$2x-3y-6z=6$或$6x+3y-2z=18$。
		\rightgrade{10}
	\end{Answer}

	\question{
		求微分方程$y''+2y'-3y=0$的一条积分曲线,使其在原点处与直线$y=4x$相切。
	}
	\begin{Answer}
		方程的通解为
		$$y=C_1e^x+C_2e^{-3x}.$$
		\rightgrade{4}
		由已知$y(0)=0,\ y'(0)=4$,代入上式得
		$$C_1=1,\ C_2=-1.$$
		\rightgrade{8}
		故所求积分曲线的方程为
		$$y=e^x-e^{-3x}.$$
		\rightgrade{10}
	\end{Answer}

	\question{
		设$\ds f(x,y)=x^2+(y^2-1)\tan\sqrt{\frac xy}$,求$f_x(x,1)$。
	}
	\begin{Answer}
		$$f_x(x,1)=\lim_{\Delta x\to0}\frac{(x+\Delta x)^2-x^2}{\Delta x}=2x$$
		或
		$$f_x(x,1)=2x+(y^2-1)\left.\left(\tan\sqrt{\frac xy}\right)'_{x}\right|_{(x,1)}=2x$$
		或
		$$f_x(x,1)=x^2,\ f_x'(x,1)=2x.$$
		\rightgrade{10}
	\end{Answer}

	\question{
		求函数$z=x^2+2y^2+xy-7y+6$的极值。
	}
	\begin{Answer}
		由$\begin{cases}
			z_x=2x+y=0, \\
			z_y=4y+x-7=0,
		\end{cases}$得驻点$(-1,2)$。
		\rightgrade{5}
		$$D=\left|\begin{matrix}
			z_{xx} & z_{xy} \\
			z_{yx} & z_{yy}
		\end{matrix}\right|
		=\left|\begin{matrix}
			2 & 1 \\ 1 & 4
		\end{matrix}\right|
		=7>0,$$
		$$z_{xx}=2>0,$$
		\rightgrade{8}
		函数$z$在点$(-1,2)$处取极小值$z(-1,2)=-1$。
		\rightgrade{10}
	\end{Answer}

	\question{
		设$f(x,y)=\sqrt{x^2+y^4}$,问$f_x(0,0)$与$f_y(0,0)$是否存在?
		若存在,求其值。
	}
	\begin{Answer}
		$\ds
			\lim_{\Delta x\to0}\frac{f(\Delta x,0)-f(0,0)}{\Delta x}
			=\lim_{\Delta x\to0}\frac{|\Delta x|}{\Delta x}
		$不存在,即$f_x(0,0)$不存在。
		\rightgrade{5}
		$\ds
			\lim_{\Delta y\to0}\frac{f(0,\Delta y)-f(0,0)}{\Delta y}
			=\lim_{\Delta y\to0}\Delta y=0
		$即$f_y(0,0)=0$。
		\rightgrade{10}
	\end{Answer}

	\question{
		设函数$z=f(x,y)$满足关系式$f(tx,ty)=t^kf(x,y)$,试证$f(x,y)$能化成$\ds z=x^kF\left(\frac yx\right)$的形式。
	}
	\begin{Answer}
		在$f(tx,ty)=t^kf(x,y)$中,令$\ds t=\frac1x$,
		则$\ds f\left(1,\frac yx\right)=\frac1{x^k}f(x,y)$。
		所以$f(x,y)=x^kf\left(1,\frac yx\right)$。
		\rightgrade{6}
		令$\ds f\left(1,\frac yx\right)=F\left(\frac yx\right)$,
		所以$\ds f(x,y)=x^kF\left(\frac yx\right)$。
		\rightgrade{10}
	\end{Answer}

	\question{
		求欧拉方程$x^2y''+xy'-9y=0$的通解。
	}
	\begin{Answer}
		令$x=e^t$做换元,
		\rightgrade{2}
		原方程变为
		$$\D(\D-1)y+\D y-9y=0,$$
		即
		\begin{equation}
			\D^2y-9y=0,
			\label{eq:1}
		\end{equation}
		即
		$$\derivative[2]{t}{y}-9y=0.$$
		\rightgrade{5}
		这是常系数线性方程,其特征方程为:$r^2-9=0$。
		所以,$r=\pm3$。
		所以方程\eqref{eq:1}的通解为:
		$$y=C_1e^{3t}+C_2e^{-3t},$$
		$C_1,\ C_2$任意。
		将$t=\ln x$代入,得欧拉方程的通解为:
		$$y=C_1x^3+C_2x^{-3},$$
		$C_1,\ C_2$任意。
		\rightgrade{10}
	\end{Answer}
	
	\question{
		求函数$z=x^2-xy+y^2$在点$(1,1)$处沿单位矢量$\vec l=\langle\cos\alpha,\sin\alpha\rangle$方向的方向导数,并求$\alpha$分别取什么值时,沿$\vec l$方向的方向导数最大,最小或等于零。
	}
	\begin{Answer}
		$$
			\left.\partial xz\right|_{(1,1)}=\left.(2x-y)\right|_(1,1)=1,
			\quad
			\left.\partial yz\right|_{(1,1)}=\left.(2y-x)\right|_(1,1)=1,
		$$
		$$
			\partial lz=\cos\alpha+\sin\alpha
			=\langle1,1\rangle\cdot\langle\cos\alpha,\sin\alpha\rangle
			=\sqrt2\cos\varphi,
		$$
		其中$\varphi$为$\vec l=\langle\cos\alpha,\sin\alpha\rangle$与$\vec g=\langle1,1\rangle$的夹角。
		\rightgrade{5}
		所以$\varphi=0$时,即$\ds\alpha=\frac\pi4$时,$\ds\partial lz=\sqrt2$取最大值;\newline
		$\varphi=\pi$时,即$\ds\alpha=\frac{5\pi}4$时,$\ds\partial lz=-\sqrt2$取最小值;\newline
		$\ds\varphi=\frac\pi2$时,即$\alpha=\frac{3\pi}4$或$-\frac\pi4$时,$\ds\partial lz=0$。
		\rightgrade{10}
	\end{Answer}
\end{ExamPaper}